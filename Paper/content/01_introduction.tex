\section{Einleitung}
\label{sec_introduction}

\IEEEPARstart{A}{ufgrund} der breiten Verfügbarkeit von Netzwerkinfrastruktur und dem so möglichen Zugang zu IT-Ressourcen über diese Netzwerke wird das Betriebsmodell vieler Systeme geändert. Das Cloud Computing wird laut NIST durch fünf wesentliche Bestandteile definiert - die bereits erwähnte breite Verfügbarkeit von Netzwerkinfrastruktur sowie der Zugriff darauf, die schnelle und einfache Skalierbarkeit der Leistung und Reichweite von Rechenressourcen, das Zusammenfassen von Pools ebendieser Ressourcen die mehreren Nutzern zur Verfügung gestellt werden, sowie die Nutzbarkeit nach dem eigenem Bedarf und schließlich die verbrauchsabhängige Abrechnung der Ressourcen.

Dabei wird durch das NIST wie auch im allgemeinen Gebrauch in vier verschiedene Service-Modelle im Cloud Computing in Infrastructure as a Service, Platform as a Service und Software as a Service unterteilt. Die Ausprägungen dieser Service Modelle sind jedoch mitunter abhängig vom Anbieter.

Zudem können diese Dienste auf unterschiedliche Weisen abgerufen werden, die Unterscheidung in Private und Public Cloud stellt hierbei eine gebräuchliche Unterteilung dar, auch an diesem Punkt gibt es je nach Anbieter unterschiedliche Auffassungen bei der Einteilung der bereitgestellten Lösungen. Durch das NIST erfolgt so z.B. neben der Unterscheidung in Private und Public Cloud auch noch die Einteilung in Community und Hybrid Cloud. \cite{nistStandards}

Mit der zunehmenden Anzahl der Anbieter von Cloud Diensten entstehen auch immer neue Möglichkeiten Cloud-Dienste zu kreieren und darauf zuzugreifen. So könnte beispielsweise durch die Möglichkeit dedizierte virtuelle CPUs zu nutzen, die Public Cloud für den Betreiber einer Private Cloud interessanter werden. In dieser Arbeit sollen die speziellen Anforderungen an die jeweilige Serviceform erörtert und daraus resultierende Vorteile beim Betrieb in einer Private, respektive Public, Cloud aufgezeigt werden um eine zeitgemäße Empfehlung bei der Entscheidung für oder gegen das jeweilige Betriebsmodell zu geben. 

Um die verschiedenen Anbieter und deren Dienstangebot zu betrachten werden zunächst die verschiedenen Service Modelle definiert sowie die zugehörigen Bereitstellungsformen. Im Folgenden werden exemplarisch die Cloud-Angebote von Fujitsu, SAP und Amazon mit Blick auf ihre angebotenen Dienste betrachtet. Der darauffolgende Abschnitt stellt derzeit im Einsatz befindliche Anwendungen vor und in dem darauf Folgenden wird auf die heutige Verwendung der Cloud-Technologie in Unternehmen eingegangen, bevor verwandte Arbeiten betrachtet und die Ergebnisse zusammengefasst werden.
