\section{Microsoft Azure}
\label{sec_azure}

\IEEEPARstart{M}{icrosoft} Azure\footnote{\url{http://azure.microsoft.com/de-de/}} ist die Cloud-Plattform von Microsoft. 
Sie ist seit Februar 2010 verfügbar\cite{azureAvailable} und ist mit 17 Standorten weltweit verfügbar\cite{azureRegions}. 

\subsection{Cloud-Modelle}
\label{sec_azure_models}

Azure ist eine Sammlung integrierter Dienste und umfasst sowohl Angebote zu Infrastructure as a Service als auch Platform as a Service\cite{azureWhat}. 
Somit bietet Azure einerseits Dienste, die im Sinne von IaaS nicht verwaltet werden und somit höhere Flexibilität und Kontrolle bieten, als auch solche, die gemäß PaaS durch Azure verwaltet werden, durch einen höheren Abstraktionsgrad gekennzeichnet sind und somit mit reduzierten Kosten, verringerter Komplexität, sowie verbesserter Skalierbarkeit und Verfügbarkeit einhergehen\cite{azureVsAmazon}. 

In Bezug auf Betriebsmodelle ermöglicht Azure nicht die feste Auswahl zwischen Private oder Public Cloud, sondern bietet ausschließlich eine Hybride Lösung an\cite{azureVsAmazon}. 
Diese ermöglicht es, je nach Bedarf sowohl Ressourcen von anwendereigenen Rechenzentren wie auch der Public Cloud zu verwenden. 

\subsection{Weitere Eigenschaften}
\label{sec_azure_features}

Azure unterstützt mehrere Betriebssysteme und Programmiersprachen. 
So kann beispielsweise nicht nur Windows, sondern auch Linux verwendet werden. 
Kosten für die skalierbare Nutzung der Dienste werden im Minutentakt erhoben\cite{azureWhat}.

Desweiteren legt Azure besonderen Wert auf die Sicherheit der Cloud-Lösungen. 
So werden zum Beispiel sowohl Kommunikation als auch Daten verschlüsselt. 
Zudem werden Penetrationstests eingesetzt, um die Sicherheitsmaßnahmen auf die Probe zu stellen\cite{azureSecurity}.

Auch die Problematik des Datenschutzes wird berücksichtigt. 
So wird zum Beispiel garantiert, dass Daten nur in Rechenzentren innerhalb bestimmter Regionen ausgetauscht werden, sodass etwa Daten aus Europa nicht in die Vereinigten Staaten transportiert werden dürfen. 
Zudem wird das Safe Harbor Framework umgesetzt, um dem europäischen Recht hinsichtlich des geschäftlichen Datenverkehrs zu genügen\cite{azurePrivacy}.