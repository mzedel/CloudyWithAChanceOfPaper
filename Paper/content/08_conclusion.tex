\section{Ergebnisse und Ausblick}
\label{sec_conclusion}

\IEEEPARstart{I}{n} diesem Papier haben wir die verschiedenen Modelle und Umsetzungen im Bereich des Cloud Computing erörtert. 
Wir haben zuerst einen allgemeinen Überblick über die Service-Modelle, also insbesondere Infrastructure as a Service, Platform as a Service und Software as a Service, sowie deren jeweilige Vor- und Nachteile gegeben. 
Anschließend sind wir in gleicher Weise auf Betriebsmodelle im Cloud Computing, namentlich Public Cloud, Private Cloud, Hybrid Cloud und entsprechende Variationen, eingegangen. 

Sodann haben wir die Positionen ausgewählter Anbieter, namentlich SAP, Fujitsu und Amazon, zum Cloud Computing dargelegt.
% - SAP ist hardwareferner, Amazon ist hardwarenäher
% - SAP-Lösungen auf anderen Anbietern betreiben 
% - genannte Probleme in SAP- / Fujitsu-Vorträgen
% - Amazon ist dominant
% - mit der Einhaltung von Schnittstellen nicht gut dabei

...

Weiterhin haben wir den aktuellen Stand der Technik betrachtet.
% - Private Clouds sind zwecks Datensicherheit immer noch zu empfehlen und wird dementsprechend von Unternehmen gewählt
% - Interoperabilität wird durch Google und Microsoft gegen Amazon vorangetrieben
%	- insbesondere bei Docker: alle einheitliche Formate
%	- Docker gewinnt dadurch Popularität, ist bei überschneidenden Anwendungsszenarien ggü. OpenStack im Vorteil
% - Schnittstellenkompatibilität als Marktvorteil

...

% Aus Related Work:
% - viele Arbeiten zu Vergleichen von Anbietern
% - über die Jahre hinweg und je nach Methodologie verschiedene Kriterien zur Einschätzung von Cloud-Anbietern
% - aber: egal ob Virtualisierung oder schlichter Speicherplatz:
%	- es gibt verschiedene Kriterien
%	- kein Anbieter ist überall perfekt
%	- Benutzer hat bestimmte Ansprüche
%	- Benutzer muss anhand seiner Ansprüche abwägen und geeigneten Anbieter wählen
% - bei Speicherplatzanbietervergleich: Unterstützung von mobilen Geräten berücksichtigt
%	-> technischer Fortschritt ist relevant
% - Schnittstellenkompatibilität -> gut für Cloud-Interoperabilität

Wir haben zudem einige der vielen Arbeiten genannt, welche sich mit dem Vergleich von Cloud-Anbietern beschäftigen. 
Es ist ersichtlich, dass sich die Kriterien und Methodiken zur Beurteilung von Anbietern über die Zeit hinweg verändert haben. 
Auch werden nicht nur Anbieter von Cloud Computing, sondern auch von Speicherplatz in der Cloud verglichen. 
Eine Arbeit zum Vergleich von Speicherplatzanbietern hat zudem auch die Unterstützung mobiler Geräte berücksichtigt, was als Beispiel dafür dienen mag, dass der allgemeine technische Fortschritt die Ansprüche, welche an Cloud-Technologien gestellt werden, beeinflusst. 
Insgesamt wurde von den jeweiligen Autoren stets betont, dass die Qualität der Leistung eines Anbieters verschiedene Aspekte umfasst und ein Benutzer individuelle Ansprüche hat, anhand derer er nach entsprechender Abwägung den geeigneten Anbieter auswählen muss. 
Die Einschätzung der Qualität von Cloud-Anbietern und deren Vergleich wird sicherlich auch in der Zukunft von großer Relevanz sein. 

% Abschluss
% - es gab und gibt verschiedene, insgesamt dominante Anbieter
% - verschiedene Lösungen
% - Anwender muss entsprechend seiner Anforderungen entscheiden
% - schauen, wie sich Trends und Angebote im Zusammenspiel mit technologischer Weiterenwicklung verändern
% - irgendein wolkiger Abschlusssatz :)

Zusammenfassend kann gesagt werden, dass es verschiedene Anbieter und Angebote im Bereich des Cloud Computing gab und gibt. 
Es gibt bislang keine allumfassende, einzig wahre Lösung für sämtliche Anforderungen an das Cloud Computing, stattdessen müssen Anwender je nach Anwendungsfall entscheiden. 
Es bleibt abzuwarten, wie sich Trends und Angebote im Zusammenspiel mit dem technischen Fortschritt weiterentwickeln. 
Die Zukunft des Cloud Computing steht in den Wolken.