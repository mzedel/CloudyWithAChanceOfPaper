\section{Ergebnisse und Ausblick}
\label{sec_conclusion}

\IEEEPARstart{I}{n} diesem Papier haben wir die verschiedenen Modelle und Umsetzungen im Bereich des Cloud Computing erörtert. 
Wir haben zuerst einen allgemeinen Überblick über die Service-Modelle, also insbesondere Infrastructure as a Service, Platform as a Service und Software as a Service, sowie deren jeweilige Vor- und Nachteile gegeben. 
Anschließend sind wir in gleicher Weise auf Betriebsmodelle im Cloud Computing, namentlich Public Cloud, Private Cloud, Hybrid Cloud und entsprechende Variationen, eingegangen. 

% Anbieter:
% - Fujitsu-Vortrag-Probleme
% - Amazon ist dominant
% - mit der Einhaltung von Schnittstellen nicht gut dabei

Sodann haben wir die Positionen und Angebote ausgewählter Anbieter, namentlich Fujitsu, Microsoft und Amazon, zum Cloud Computing dargelegt. 
Es wurden einige Herausforderungen hinsichtlich des Cloud Computing identifiziert, welche von verschiedenen Angeboten in unterschiedlichem Maß berücksichtigt werden. 
So bietet Fujitsu beispielsweise sowie Public wie auch Private Cloud Lösungen an, wobei erstere durch bessere Skalierbarkeit und letztere durch höhere Flexibilität gekennzeichnet sind. 
Microsoft Azure unterstützt aus ähnlichen Gründen sowohl IaaS als auch PaaS und wirkt zudem Bedenken hinsichtlich Datensicherheit und Datenschutz entgegen. 
Wie auch Azure stellt Amazon eine Vielzahl von Diensten bereit und ist zur Zeit dominanter Anbieter.

% Stand der Technik:
% - Private Clouds sind zwecks Datensicherheit immer noch zu empfehlen und wird dementsprechend von Unternehmen gewählt
% - Interoperabilität wird durch Google und Microsoft gegen Amazon vorangetrieben
%	- insbesondere bei Docker: alle einheitliche Formate
%	- Docker gewinnt dadurch Popularität, ist bei überschneidenden Anwendungsszenarien ggü. OpenStack im Vorteil
% - Schnittstellenkompatibilität als Marktvorteil

Weiterhin haben wir den aktuellen Stand der Technik betrachtet.
Es zeigt sich, dass Unternehmen noch immer häufig zu Private Clouds tendieren, um Bedenken hinsichtlich der Datensicherheit Rechnung zu tragen. 
Im Bezug auf Interoperabilität tun sich Google und Microsoft hervor, die sich gegenüber Amazon einen Marktvorteil durch Umsetzung einheitlicher Schnittstellen erhoffen. 
Diese Einheitlichkeit ist auch für Docker von Vorteil, welches sich dadurch in den entsprechenden Anwendungsgebieten von anderen Lösungen wie OpenStack abheben kann. 

% Aus Related Work:
% - viele Arbeiten zu Vergleichen von Anbietern
% - über die Jahre hinweg und je nach Methodologie verschiedene Kriterien zur Einschätzung von Cloud-Anbietern
% - aber: egal ob Virtualisierung oder schlichter Speicherplatz:
%	- es gibt verschiedene Kriterien
%	- kein Anbieter ist überall perfekt
%	- Benutzer hat bestimmte Ansprüche
%	- Benutzer muss anhand seiner Ansprüche abwägen und geeigneten Anbieter wählen
% - bei Speicherplatzanbietervergleich: Unterstützung von mobilen Geräten berücksichtigt
%	-> technischer Fortschritt ist relevant
% - Schnittstellenkompatibilität -> gut für Cloud-Interoperabilität

Wir haben zudem einige der vielen Arbeiten genannt, welche sich mit dem Vergleich von Cloud-Anbietern beschäftigen. 
Es ist ersichtlich, dass sich die Kriterien und Methodiken zur Beurteilung von Anbietern über die Zeit hinweg verändert haben. 
Einige der zuvor genannten Herausforderungen beim Cloud Computing, insbesondere Skalierbarkeit und Datensicherheit, finden sich auch als Kriterien in den genannten Arbeiten wieder. 

Eine Arbeit zum Vergleich von Speicherplatzanbietern hat zudem auch die Unterstützung mobiler Geräte berücksichtigt, was als Beispiel dafür dienen mag, dass der allgemeine technische Fortschritt die Ansprüche, welche an Cloud-Technologien gestellt werden, beeinflusst. 
Fujitsu berücksichtigt solche Geräte in seinem SaaS-Portfolio ebenso.

Auch besteht im Zusammenhang mit der Interoperabilität von Cloud-Anwendungen über mehrere Anbieter hinweg der Bedarf nach der Vereinheitlichung von Schnittstellen, was auch hinsichtlich der zuvor erwähnten Bestrebungen seitens Google und Microsoft von Interesse für die Zukunft des Cloud Computing sein könnte.

In vielen Arbeiten wurde betont, dass die Qualität der Leistung eines Anbieters verschiedene Aspekte umfasst und ein Benutzer individuelle Ansprüche hat, anhand derer er nach entsprechender Abwägung den geeigneten Anbieter auswählen muss. 
Die Einschätzung der Qualität von Cloud-Anbietern und deren Vergleich wird also  sicherlich auch in der Zukunft von großer Relevanz sein. 

% Abschluss
% - es gab und gibt verschiedene, insgesamt dominante Anbieter
% - verschiedene Lösungen
% - Anwender muss entsprechend seiner Anforderungen entscheiden
% - schauen, wie sich Trends und Angebote im Zusammenspiel mit technologischer Weiterenwicklung und Kompatibilitätsbedürfnis verändern
% - irgendein wolkiger Abschlusssatz :)

Zusammenfassend kann gesagt werden, dass es sehr unterschiedliche Anbieter und Angebote im Bereich des Cloud Computing gab und gibt. 
Es gibt bislang keine allumfassende, einzig wahre Lösung für sämtliche mögliche Anforderungen, stattdessen müssen Anwender sich je nach Anwendungsfall für eine passende Umsetzung entscheiden. 
Es bleibt abzuwarten, wie sich Trends und Angebote im Zusammenspiel mit dem technischen Fortschritt und dem Bedürfnis nach Kompatibilität weiterentwickeln. 
Die Zukunft des Cloud Computing steht somit in den Wolken.
