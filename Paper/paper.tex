% 
% Tim Sporleder, Manuel Zedel
%
% Ausarbeitung für das Seminar
%	Industrie-Seminar: "Cloud Computing - Wie die Industrie sich neuen Herausforderungen stellt"
%	von Prof. Dr. Andreas Polze
%	Hasso-Plattner-Institut Potsdam
%	Wintersemester 2014/2015
%
% Basiert auf:
% 	bare_jrnl.tex
% 	V1.4a
% 	2014/09/17
% 	by Michael Shell
%

\documentclass[journal]{IEEEtran}

\usepackage{cite}

\usepackage[pdftex]{graphicx}
\graphicspath{{images/}}
\DeclareGraphicsExtensions{.pdf,.jpeg,.png}

\usepackage[cmex10]{amsmath}
\interdisplaylinepenalty=2500

\usepackage{algorithmic}

\usepackage{array}

\usepackage[caption=false,font=footnotesize]{subfig}

\usepackage{fixltx2e}

\usepackage{url}
\usepackage[utf8]{inputenc} 
\usepackage[ngerman]{babel} 

\hyphenation{op-tical net-works semi-conduc-tor}

\begin{document}

\title{Kartoffelsuppe in der Cloud - IaaS, PaaS und SaaS aus der privaten Mensa für grandioses Aroma!}

\author{Tim~Sporleder und Manuel~Zedel\thanks{}}

% The paper headers
\markboth{Journal of \LaTeX\ Class Files,~Vol.~13, No.~9, September~2014}%
{Shell \MakeLowercase{\textit{et al.}}: Bare Demo of IEEEtran.cls for Journals}
% The only time the second header will appear is for the odd numbered pages
% after the title page when using the twoside option.
% 
% *** Note that you probably will NOT want to include the author's ***
% *** name in the headers of peer review papers.                   ***
% You can use \ifCLASSOPTIONpeerreview for conditional compilation here if
% you desire.




% If you want to put a publisher's ID mark on the page you can do it like
% this:
%\IEEEpubid{0000--0000/00\$00.00~\copyright~2014 IEEE}
% Remember, if you use this you must call \IEEEpubidadjcol in the second
% column for its text to clear the IEEEpubid mark.



% use for special paper notices
%\IEEEspecialpapernotice{(Invited Paper)}




% make the title area
\maketitle

% As a general rule, do not put math, special symbols or citations
% in the abstract or keywords.
\begin{abstract}
The abstract goes here.
\end{abstract}

% Note that keywords are not normally used for peerreview papers.
\begin{IEEEkeywords}
IEEEtran, journal, \LaTeX, paper, template.
\end{IEEEkeywords}






% For peer review papers, you can put extra information on the cover
% page as needed:
% \ifCLASSOPTIONpeerreview
% \begin{center} \bfseries EDICS Category: 3-BBND \end{center}
% \fi
%
% For peerreview papers, this IEEEtran command inserts a page break and
% creates the second title. It will be ignored for other modes.
\IEEEpeerreviewmaketitle



\section{Einleitung}
% The very first letter is a 2 line initial drop letter followed
% by the rest of the first word in caps.
% 
% form to use if the first word consists of a single letter:
% \IEEEPARstart{A}{demo} file is ....
% 
% form to use if you need the single drop letter followed by
% normal text (unknown if ever used by IEEE):
% \IEEEPARstart{A}{}demo file is ....
% 
% Some journals put the first two words in caps:
% \IEEEPARstart{T}{his demo} file is ....
% 
% Here we have the typical use of a "T" for an initial drop letter
% and "HIS" in caps to complete the first word.
\IEEEPARstart{A}{ufgrund} der breiten Verfügbarkeit von Netzwerkinfrastruktur und dem so möglichen Zugang zu IT-Ressourcen über diese Netzwerke wird das Betriebsmodell vieler Systeme geändert. Das Cloud Computing wird laut NIST durch fünf wesentliche Bestandteile definiert - die bereits erwähnte breite Verfügbarkeit von Netzwerkinfrastruktur sowie der Zugriff darauf, die schnelle und einfache Skalierbarkeit der Leistung und Reichweite von Rechenressourcen, das Zusammenfassen von Pools ebendieser Ressourcen, sowie die Nutzbarkeit nach dem eigenem Bedarf und schließlich die Verbrauchs-gemäße Abrechnung der Ressourcen.

Dabei wird durch das NIST wie auch im allgemeinen Gebrauch in vier verschiedene Service Modelle im Cloud Computing in Infrastructure as a Service, Platform as a Service und Software as a Service unterteilt. Die Ausprägungen dieser Service Modelle sind jedoch mitunter Anbieter-spezifisch.

Zudem können diese Dienste auf unterschiedliche Weisen abgerufen werden, die Unterscheidung in Private und Public Cloud stellt hierbei eine gebräuchliche Unterteilung dar, auch an diesem Punkt gibt es Anbieter-spezifische Auffassungen bei der Einteilung der bereitgestellten Lösungen. Durch das NIST erfolgt so z.B. neben der Unterscheidung in Private und Public Cloud auch noch die Einteilung in Community und Hybrid Cloud.

Mit der zunehmenden Anzahl der Anbieter von Cloud Diensten entstehen auch immer neue Möglichkeiten Cloud-Dienste zu kreieren und darauf zuzugreifen. So könnte beispielsweise durch die Möglichkeit dedizierte virtuelle CPUs zu nutzen, die Public Cloud für den Betreiber einer Private Cloud interessanter werden. In dieser Arbeit sollen die speziellen Anforderungen an die jeweilige Serviceform erörtert und daraus resultierende Vorteile beim Betrieb in einer Private, respektive Public, Cloud aufgezeigt werden um eine zeitgemäße Empfehlung bei der Entscheidung für oder gegen das jeweilige Betriebsmodell zu geben. 

Um die verschiedenen Service Modelle genauer zu untersuchen sollen sie zunächst definiert werden und exemplarisch die Unterschiede bei dieser Definition zwischen verschiedenen Herstellern aufgezeigt. Im Anschluss erfolgt die selbige Herangehensweise zur Definition der Eigenschaften von Private und Public Clouds sowie die weiteren Ausprägungsformen der Betriebsmodelle. Auf dieser Grundlage werden die verschiedenen Service Modelle unter Beachtung typischer Einsatzzwecke auf die Möglichkeit der Nutzung im Rahmen der verschiedenen Betriebsmodelle überprüft.

\hfill 27. Februar, 2014

\subsection{Subsection Heading Here}
Subsection text here.

% needed in second column of first page if using \IEEEpubid
%\IEEEpubidadjcol

\subsubsection{Subsubsection Heading Here}
Subsubsection text here.


\section{$\{I|P|S\}$aaS}
\IEEEPARstart{Ü}{}blicherweise wird Cloud Computing als Schichtenmodell beschrieben, bei dem Schichten durch die Service Modelle charakterisiert werden. Die unterste Schicht bildet die Infrastruktur der Lösung, bestehend aus der zum Betrieb nötigen Hard- und Software. Die darüber liegende Schicht wird als Plattform bezeichnet, sie besteht für gewöhnlich aus einer Sammlung an Werkzeugen und Diensten mit denen das Programmieren und Bereitstellen neuer Cloudanwendungen vereinfacht und beschleunigt wird. Auf diesen beiden Schichten befindet sich die Software Schicht, mit der Endnutzer in Form von bestehenden Webapplikationen interagieren.

Auch wenn diese Schichten nicht zusammen genutzt werden müssen, erschließt sich der Zusammenhang jedoch leicht anhand einer Analogie mit der Berliner S-Bahn: Die Schienen bilden die Infrastruktur, die darauf befindlichen Züge stellen die Plattform dar und das S-Bahn Personal, welches die Züge betreibt und so Fahrgäste transportiert dient als Software, welche die ihr zugewiesenen Schienen und Züge nutzt. Im Fall eines Schienenbruchs könnten die Züge auf anderen Strecken zum Einsatz kommen und das Personal könnte auf anderen Verbindungen eingesetzt werden um Fahrgäste zu transportieren. 

Dabei gilt es zu beachten, dass die Grenzen zwischen den einzelnen Schichten nicht immer klar gezogen werden können, analog könnte die S-Bahn möglicherweise eine speziell angepasste Antriebstechnik benötigen und so direkt an die Berliner Infrastruktur gekoppelt sein, oder nach der Umstellung auf Draisinen eine der Infrastruktur nähere Betriebsweise eingeführt haben - wodurch der Plattform Charakter in den Hintergrund gerückt werden würde.

Im folgenden Teil der Arbeit werden die verschiedenen Schichten bzw. Service Modelle vorgestellt und auf ebenfalls entstandene Mischformen eingegangen.
 
\subsection{Infrastructure as a Service}
Subsection text here.

\subsection{Platform as a Service}
Subsection text here.

\subsection{Software as a Service}
Subsection text here.

\subsection{Whatever as a Service}
und dazwischen gibts auch noch:



\section{Private vs. Public}
All my clouds are belong to you!

\subsection{Private Cloud}
Subsection text here.

\subsection{Public Cloud}
Subsection text here.

\subsection{Hybrid Cloud}
Subsection text here.



\section{Private vs. Publicly as a Service}
All my services are belong to you!

\subsection{Whatever as a Service auf meinem PC in der VM bei Goomazoft}
ich schiebe meine VM dank OpenStack zur Konkurrenz und bin trotzdem der King of my Castle


% An example of a floating figure using the graphicx package.
% Note that \label must occur AFTER (or within) \caption.
% For figures, \caption should occur after the \includegraphics.
% Note that IEEEtran v1.7 and later has special internal code that
% is designed to preserve the operation of \label within \caption
% even when the captionsoff option is in effect. However, because
% of issues like this, it may be the safest practice to put all your
% \label just after \caption rather than within \caption{}.
%
% Reminder: the "draftcls" or "draftclsnofoot", not "draft", class
% option should be used if it is desired that the figures are to be
% displayed while in draft mode.
%
%\begin{figure}[!t]
%\centering
%\includegraphics[width=2.5in]{myfigure}
% where an .eps filename suffix will be assumed under latex, 
% and a .pdf suffix will be assumed for pdflatex; or what has been declared
% via \DeclareGraphicsExtensions.
%\caption{Simulation results for the network.}
%\label{fig_sim}
%\end{figure}

% Note that IEEE typically puts floats only at the top, even when this
% results in a large percentage of a column being occupied by floats.


% An example of a double column floating figure using two subfigures.
% (The subfig.sty package must be loaded for this to work.)
% The subfigure \label commands are set within each subfloat command,
% and the \label for the overall figure must come after \caption.
% \hfil is used as a separator to get equal spacing.
% Watch out that the combined width of all the subfigures on a 
% line do not exceed the text width or a line break will occur.
%
%\begin{figure*}[!t]
%\centering
%\subfloat[Case I]{\includegraphics[width=2.5in]{box}%
%\label{fig_first_case}}
%\hfil
%\subfloat[Case II]{\includegraphics[width=2.5in]{box}%
%\label{fig_second_case}}
%\caption{Simulation results for the network.}
%\label{fig_sim}
%\end{figure*}
%
% Note that often IEEE papers with subfigures do not employ subfigure
% captions (using the optional argument to \subfloat[]), but instead will
% reference/describe all of them (a), (b), etc., within the main caption.
% Be aware that for subfig.sty to generate the (a), (b), etc., subfigure
% labels, the optional argument to \subfloat must be present. If a
% subcaption is not desired, just leave its contents blank,
% e.g., \subfloat[].


% An example of a floating table. Note that, for IEEE style tables, the
% \caption command should come BEFORE the table and, given that table
% captions serve much like titles, are usually capitalized except for words
% such as a, an, and, as, at, but, by, for, in, nor, of, on, or, the, to
% and up, which are usually not capitalized unless they are the first or
% last word of the caption. Table text will default to \footnotesize as
% IEEE normally uses this smaller font for tables.
% The \label must come after \caption as always.
%
%\begin{table}[!t]
%% increase table row spacing, adjust to taste
%\renewcommand{\arraystretch}{1.3}
% if using array.sty, it might be a good idea to tweak the value of
% \extrarowheight as needed to properly center the text within the cells
%\caption{An Example of a Table}
%\label{table_example}
%\centering
%% Some packages, such as MDW tools, offer better commands for making tables
%% than the plain LaTeX2e tabular which is used here.
%\begin{tabular}{|c||c|}
%\hline
%One & Two\\
%\hline
%Three & Four\\
%\hline
%\end{tabular}
%\end{table}


% Note that the IEEE does not put floats in the very first column
% - or typically anywhere on the first page for that matter. Also,
% in-text middle ("here") positioning is typically not used, but it
% is allowed and encouraged for Computer Society conferences (but
% not Computer Society journals). Most IEEE journals/conferences use
% top floats exclusively. 
% Note that, LaTeX2e, unlike IEEE journals/conferences, places
% footnotes above bottom floats. This can be corrected via the
% \fnbelowfloat command of the stfloats package.


\section{Verwandte Arbeiten}
Bing
SAP
Fujitsu
HP
Kontext

\section{Zusammenfassung}
Cloud.. mit Kartoffeln, als Suppe, aus dem Topf - Junge!



% if have a single appendix:
%\appendix[Proof of the Zonklar Equations]
% or
%\appendix  % for no appendix heading
% do not use \section anymore after \appendix, only \section*
% is possibly needed

% use appendices with more than one appendix
% then use \section to start each appendix
% you must declare a \section before using any
% \subsection or using \label (\appendices by itself
% starts a section numbered zero.)
%


\appendices
\section{Proof of the First Zonklar Equation}
Appendix one text goes here.

% you can choose not to have a title for an appendix
% if you want by leaving the argument blank
\section{}
Appendix two text goes here.


% use section* for acknowledgment
\section*{Acknowledgment}


The authors would like to thank...


% Can use something like this to put references on a page
% by themselves when using endfloat and the captionsoff option.
\ifCLASSOPTIONcaptionsoff
  \newpage
\fi



% trigger a \newpage just before the given reference
% number - used to balance the columns on the last page
% adjust value as needed - may need to be readjusted if
% the document is modified later
%\IEEEtriggeratref{8}
% The "triggered" command can be changed if desired:
%\IEEEtriggercmd{\enlargethispage{-5in}}

% references section

% can use a bibliography generated by BibTeX as a .bbl file
% BibTeX documentation can be easily obtained at:
% http://www.ctan.org/tex-archive/biblio/bibtex/contrib/doc/
% The IEEEtran BibTeX style support page is at:
% http://www.michaelshell.org/tex/ieeetran/bibtex/
%\bibliographystyle{IEEEtran}
% argument is your BibTeX string definitions and bibliography database(s)
%\bibliography{IEEEabrv,../bib/paper}
%
% <OR> manually copy in the resultant .bbl file
% set second argument of \begin to the number of references
% (used to reserve space for the reference number labels box)
\begin{thebibliography}{1}

\bibitem{IEEEhowto:kopka}
H.~Kopka and P.~W. Daly, \emph{A Guide to \LaTeX}, 3rd~ed.\hskip 1em plus
  0.5em minus 0.4em\relax Harlow, England: Addison-Wesley, 1999.

\end{thebibliography}

% biography section
% 
% If you have an EPS/PDF photo (graphicx package needed) extra braces are
% needed around the contents of the optional argument to biography to prevent
% the LaTeX parser from getting confused when it sees the complicated
% \includegraphics command within an optional argument. (You could create
% your own custom macro containing the \includegraphics command to make things
% simpler here.)
%\begin{IEEEbiography}[{\includegraphics[width=1in,height=1.25in,clip,keepaspectratio]{mshell}}]{Michael Shell}
% or if you just want to reserve a space for a photo:

\begin{IEEEbiography}{Michael Shell}
Biography text here.
\end{IEEEbiography}

% if you will not have a photo at all:
\begin{IEEEbiographynophoto}{John Doe}
Biography text here.
\end{IEEEbiographynophoto}

% insert where needed to balance the two columns on the last page with
% biographies
%\newpage

\begin{IEEEbiographynophoto}{Jane Doe}
Biography text here.
\end{IEEEbiographynophoto}

% You can push biographies down or up by placing
% a \vfill before or after them. The appropriate
% use of \vfill depends on what kind of text is
% on the last page and whether or not the columns
% are being equalized.

%\vfill

% Can be used to pull up biographies so that the bottom of the last one
% is flush with the other column.
%\enlargethispage{-5in}



% that's all folks
\end{document}


